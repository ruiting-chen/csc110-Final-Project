\documentclass[fontsize=11pt]{article}
\usepackage{amsmath}
\usepackage[utf8]{inputenc}
\usepackage[margin=0.75in]{geometry}
\usepackage{graphicx}

\title{CSC110 Project Proposal: From Sea Level Raising to Global Warming: Analysing the Relationship Between Sea Level and Climate Change}
\author{Jiankun Wei, Ruiting Chen, Yiteng Sun, Weiheng Wang}
\date{Friday, November 6, 2020}

\begin{document}
    \maketitle

    \section*{Problem Description and Research Question}

    Since the industrial revolution, humans have altered the world greatly from polluting the environment to destroying the habitat. We uninhibitedly release greenhouse gases into the air causing global warming, leaving heavy metals around mines to wipe out local plants, deforest Amazon for farming, and gazing. ("Deforestation in Brazilian Amazon surges to 12-year high") As a result, the species extinction rate of the world has soared from one species per decade to six thousand species per year, which has increased sixty thousand-fold! Yet in the past, humans have taken this for granted. We indeed alter the environment and many species go extinct, but it does not affect us. In fact, our metropolis mushroomed all around the world; airplanes flying high above the sky carrying people to explore previously unreachable places; factories had been built across the globe sending manufactured goods to everybody to improve our life. Even the moon, which our ancestors can only fancy about, has been reached. It seems all fine. But does it?\\
    Gradually and unnoticeable by the proud humans, the Earth is altering and a great danger appears on the horizon -- the Raising of the sea level. ("This is what climate change look like") There are two main reasons for raising the sea level, all of them are due to the increasing amount of greenhouse gases in the atmosphere. As there are more and more greenhouse gases released into the air, it begins trapping sunlight which increases the temperature of the Earth. This in turn melts the ice on the north and south pole. The melted water flows into the sea, which raises the sea level. (“Why Are Glaciers and Sea Ice Melting?”) Also, as temperature rises, water tends to expand, directly leading to increased sea levels. Finally, as the sea level increases, more and more ice on the poles will be submerged underwater, and begin to melt itself. This is crucial because as the underlying ice melt, the iceberg fell into the sea which will eventually disappear resulting in a self-enhancing loop. ("NASA Sea Level Change Portal: Thermal Expansion.")\\
    The rising sea level directly affects human society. First, a devastating tsunami will visit the coastline more frequently causing thousands of lives and money. The deconstruction of shelters will also turn people into refugees. What is more, seawater will taint the coastal farms and add salt into the soil which reduces the crop yield. Last but not the least, many cities located near the sea are directly threatened by the rising sea level. Following the current trend, big metropolis like London, San Diego, New York, and countries like the Netherlands will soon be submerged underwater. (Lewis, Sophie)\\

    Given these important threatens, it is emergent to be aware and understand climate change and the rising sea level, and to treat them seriously.\\

    Our research question is "How does sea-level change around the world due to climate change?”

    \newpage
    \section*{Dataset Description}
    We gathered sea level data from the University of Hawaii Sea Level center website(UHSLC Legacy Data Portal), and temperature data from datahub. ("Global Temperature Time Series") The data downloaded from this website includes a collection of daily sea level recording (unit: mm) from different sea level stations across the world during the time span indicated in the data stored in .dat format. An example of sea level data from one station(Peter \& Paul Rocks Brazil) is shown in the image attached with this file.

    Each data set contains a header(first row of data) and the body of the sea level data(the rest of the rows).\\

    Header structure(from left to right):
    \begin{itemize}
        \item
        station index: 3 digits station number followed by one letter from A to Z
        \item
        name and region of the station
        \item
        start and end year of data collection
        \item
        latitude and longitude of the station
        \item
        decimation method, Coded as
        1 : filtered,
        2 : simple average of all hourly values for a day,
        3 : other
        \item
        reference offset: constant offset to be added to each data value for data to be relative to tide staff zero or primary datum in same units as data
        \item
        reference code: R = data referenced to datum,
        X = data not referenced to datum
        \item
        unit of measurement
    \end{itemize}

    Body structure(from left to right):
    \begin{itemize}
        \item
        station index: 3 digits station number followed by one letter from A to Z
        \item
        name and region of the station
        \item
        year of data collection
        \item
        julian day: day count from 1 Jan for 1st data point of each record followed by letter J for clarity
        \item
        12 daily sea level values(missing data indicated by(9999))
    \end{itemize}


    \section*{Computational Plan}

    In order to let the data set be convenient for a computer to use, our group will write functions to read .html format file and .json format file to let python conveniently use this file to do complicated operations and then get assess to the data we need. ("Extracting Data from HTML with BeautifulSoup") Then we will create three dataclasses named “SeaLevel”, "Temperature" and "Station" to recored and store the data we need. In the next step, based on our entities, we would create a new class named "ClimateSeaLevelSystem", which is similar to the system we learnt in the lecture, to store all the data mapping to their key element(like station name matches its corresponding Station). After this step, we could create an abstract class named "EntityGenerator" to generate the Temperature data and Station with its corresponding sea level data and store them in our system "ClimateSeaLevelSystem". Then because the data type in our new data set is the continuous variables in the quantitative data type (Chopra \& Alaudeen(2019) page 1), thus we decide to use linear regression, and the scatter plots on map to present data by using plotly.\\
    First, we want to present the relationship between the year (x-axis) and average daily sea-level value measured by one station (y-axis). We will first plot all the points out and then draw linear regression based on these points in the graph. What is more, we will also design a function to use linear regression to predict future changes in sea levels at different stations. \\
    Then, we also design to show the relationship between the year (x-axis) and temperature measured by one station (y-axis). The methods is the same to the steps we take in creating the linear regression of sea level. To find the relation between sea level change and climate change, we will combine two lines in one chart. \\
    Finally, we decided to use plotly to draw a world map with scatter plots about how the sea-level changes at all our stations from around 1880 to 2020, using the sea level data, station data from our data set. every single point on the map represents the actual location of each station. In the world map, we decide to use different color of plots to represent whether the sea level of the station is higher or lower than the base height(we also can say the sea level is rising or falling). For stations which reported rising sea level, we will use red to illustrate the sea level raised. For stations that reported falling sea level, we will use blue to illustrate the sea-level falls. Then as time moves forward, the colour will change, given the change in the rising and falling sea level as a vivid animation.\\
    The plotly has the ability to draw animation which makes it relevant because by using plotly to draw animation, we can have a direct visualization of the data. ("Intro to Animations") As we depict the rising and falling of the sea level as time moves forward, the animation conveys in a vivid way the data which is just what we want to study. \\
    Using plotly to draw animation is also appropriate because it reflects how sea level has changed in the recorded time period. This will give us valuable intuition on the data and also the importance of climate change. For example, when you see that as time passes, more and more stations become red, you will immediately realize the significance of climate change on our world and how less time we have. Thus, by viewing the animation, one will realize that the problem of climate change is something that we must treat seriously.


    \section*{Instructions of obtaining data sets and running our program}

    1. Install all the python libraries we listed in requirement.txt. To download BeautifulSoup4, request, datapackage, a direct way to install them is that you can open the file named "data\_process.py" and then move to these libraries for a while, it will pop up a reminder and then click installation.\\
    2. In our file named "data\_process.py", we have already provided functions to extract our datasets. Therefore, you do not need to download them from provided website. You can call the functions in this file to see what our datasets look like. \\
    3. When you finish the previous steps, you can open the file "main.py" and run it in python console. After all the temperature and selected stations are added, you can see a world map with 20 plots on it, which represent the stations. Our starting time is 1990-12. You can drag the timeline below to see the color changes of the stations. To illustrate the graph of given station, you need to type in the station name that you want to see the detailed report of in the python console (the name you type in must be the station added in the world map). Then you could see two linear regression: the blue one is the relationship between year and sea level, the red one is year and temperature anomalies. If you want to know the exact sea level or temperature of a month, move your mouse to the plot and then you can see the figure of sea level or temperature.


    \section*{Any Changes of Our Project Plan}

    Initially, we only focus on the relationship of year and sea level, trying to draw the regression line to show the climate change. However, we cannot easily concluded that sea level change around the world due to climate change. We need to illustrate the temperature changes in these years and then compare them with sea level changes. Therefore, we decided to draw two regression lines in one graph in order to compare them more intuitively.\\
    Besides, we also wanted to draw histogram to demonstrate the number of stations that report the sea level in a certain range at that certain year. After discussion, we changed our method and decided to create scatter plots on a world map to achieve this goal. In our final project, we use different color of a point to represent a station which its sea level below or above a certain height. When you drag the timeline, you can see the color change of these points. It is more intuitive and concise than histogram.


    \section*{Discussion}

    After finishing our programs and running our codes, we get the astonishing results: from December 1990 to August 2007, as the temperature rising, there were many stations turning "red", which indicated that overall sea level showed a rising trend especially in Atlantic Ocean and North Pacific Ocean. Fortunately, in the recent 13 years, several stations have returned to normal "blue" due to the fact that many countries dedicate to tackle global warning by taking several methods like eliminating the emission of $CO_2$. We also find an interesting fact that at both poles of the earth, as the temperature rising, sea level shows a falling trend, which is an abnormal phenomenon.\\
    However, there are also some drawbacks in our program:\\
    1. The running time of adding the stations to our ClimateSeaLevelSystem depends on the number of stations user wants to generate. The more stations we would like to add, the more time we cost in running this function. Therefore, we set the default number of stations generated to be 20 stations.\\
    2. The dataset we use misses some sea level data (Maybe because the equipment at the stations malfunctioned or the stations hasn't been built yet at that time), which are replaced by negative numbers. Thus we write codes to delete them. When we run the linear regression function, these points are not included in the graph. As for the animation graph, they are set to be default white colour at the time when measurement is missing so that it will not miss lead viewers.\\
    3. Because of the temperature dataset we used and the sea level dataset we used are different sources, so we do not have temperature measurement at each specific station but an overall temperature anomalies of the whole world. Thus, we use temperature anomaly of the world for each stations. \\
    4. Based on our dataset, we have daily sea level data of each station and monthly temperature data. In order to compare them more intuitively, we have to compute the average sea level in each month. \\
    5. When graphing, we restricted the time limit to pass 40 years because the first world meeting discussing about global warming was held 40 years ago. To recognize this fact and bring people's attenetion to the significance of global warming, we set the time duration of our animation graph to be 40 years.\\
    6. It is unavoidable that these stations have different established years. Some stations may establish later than our start year (1980). Therefore, different stations has different linear regression graphs.\\
    For our future exploration, we decide to show more details in our world map, like using pygame to draw an animation to show how sea level changes at our stations in selected period (from 1980 to 2020). In our world map, we use different colour(red and blue) to represent the rise or decline of the sea level. In our future assumption, we want to show more direct and intuitive information of sea level changes like using a gradient of colours to show different level of increase and decrease of sea level. For stations which showed the rising sea level, we will use a gradient from yellow to red to illustrate the intensity of how much sea level raised. For stations which showed the falling sea level, we will use a gradient from light blue to dark blue to illustrate the intensity of how much sea-level falls. Then as the time passes by, we can see how the colour of area changes and then get more intuitive details.

    \section*{References}
    \begin{enumerate}
        \item
        Nunez, Christina. “Sea Level Rise, Explained.” Sea Level Rise, Facts and Information, 27 Feb. 2019, www.nationalgeographic.com/environment/global-warming/sea-level-rise/.
        \item
        UHSLC Legacy Data Portal, uhslc.soest.hawaii.edu/data/?fd.
        \item
        Chopra, R., England, A., \& Alaudeen, M. N. (2019). Data science with Python: Combine Python with machine learning principles to discover hidden patterns in raw data. Birmingham: Packt Publishing.
        \item
        “Why Are Glaciers and Sea Ice Melting?” WWF, World Wildlife Fund, www.worldwildlife.org/pages/why-are-glaciers-and-sea-ice-melting.
        \item
        “NASA Sea Level Change Portal: Thermal Expansion.” NASA, NASA, 26 Mar. 2020.\\ sealevel.nasa.gov/understanding-sea-level/global-sea-level/thermal-expansion.
        \item
        Lewis, Sophie. “Rising Sea Levels on Track to Destroy the Homes of 300 Million People by 2050.” CBS News, CBS Interactive, 30 Oct. 2019, www.cbsnews.com/news/rising-sea-levels-on-track-to-destroy-homes-of-300-million-people-by-2050/.
        \item
        Shasta Darlington. "Deforestation in Brazilian Amazon surges to 12-year high". CNN World, CNN. 1 Dec, 2020.\\ https://edition.cnn.com/2020/12/01/americas/deforestation-brazil-amazon-bolsonaro-intl/index.html
        \item
        "This is what climate change looks like". CNN World, CNN. 11 Dec, 2020.\\
        https://edition.cnn.com/2020/12/11/world/gallery/climate-change-effects-penguins-2020-spc/index.html
        \item
        Gaurav Singhal. "Extracting Data from HTML with BeautifulSoup". Pluralsight.\\ https://www.pluralsight.com/guides/extracting-data-html-beautifulsoup
        \item
        "Intro to animations in Python". Graphing Libraries, Plotly. https://plotly.com/python/animations/
        \item
        "Global Temperature Time Series". Datahub, Datahub. https://datahub.io/core/global-temp\#python
        \item
        Hover Text and Formatting. (n.d.). Python | Plotly. https://plotly.com/python/hover-text-and-formatting/
        \item
        Multiple Axes. (n.d.). Python | Plotly. https://plotly.com/python/multiple-axes/
        \item
        World Climate Conference. (2020, August 25). Retrieved December 13, 2020, from\\
        https://en.wikipedia.org/wiki/World\_Climate\_Conference
    \end{enumerate}

% NOTE: LaTeX does have a built-in way of generating references automatically,
% but it's a bit tricky to use so we STRONGLY recommend writing your references
% manually, using a standard academic format like APA or MLA.
% (E.g., https://owl.purdue.edu/owl/research_and_citation/apa_style/apa_formatting_and_style_guide/general_format.html)

\end{document}
